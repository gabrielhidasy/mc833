% Created 2015-11-10 Tue 00:26
\documentclass[11pt]{article}
\usepackage[utf8]{inputenc}
\usepackage[T1]{fontenc}
\usepackage{fixltx2e}
\usepackage{graphicx}
\usepackage{grffile}
\usepackage{longtable}
\usepackage{wrapfig}
\usepackage{rotating}
\usepackage[normalem]{ulem}
\usepackage{amsmath}
\usepackage{textcomp}
\usepackage{amssymb}
\usepackage{capt-of}
\usepackage{hyperref}
\author{Gabriel Hidasy Rezende}
\date{\today}
\title{Atividade 7 - Estudo do backlog e tratamento de processos zumbis}
\hypersetup{
 pdfauthor={Gabriel Hidasy Rezende},
 pdftitle={},
 pdfkeywords={},
 pdfsubject={},
 pdfcreator={Emacs 24.5.1 (Org mode 8.3.2)}, 
 pdflang={English}}
\begin{document}
\maketitle
\section{Questão 1}
\label{sec:orgheadline1}
No linux, desde o kernel 2.2 o parâmetro backlog é usado como uma indicação (que
não precisa ser seguida a risca) para o tamanho da file de conexões
completamente estabelecidas (SYN-SYN/ACK-ACK) e aguardando o uso de \texttt{accept}, e
o número máximo de conexões em espera (para o qual o sistema ainda não enviou o
pacote SYN/ACK) é definido no parâmetro do kernel \texttt{tcp\_max\_syn\_backlog},
acessível através da /proc \\(\texttt{/proc/sys/net/ipv4/tcp\_max\_syn\_backlog}).

Essa maneira de interpretar o parâmetro backlog é fundamentalmente diferente de
sistemas como o BSD, onde o parâmetro é usado como limite para a soma das
conexões estabelecidas e em espera.

Teoricamente isso significaria que, no linux, após o a conexão ser estabelecida,
o sistema  colocaria imediatamente a conexão na segunda lista (que é local por
aplicação), e se a mesma estivesse cheia derrubaria a conexão (enviando um RST),
o que ele de fato faz se o parâmetro \texttt{tcp\_abort\_on\_overflow} estiver com valor 1.

Mas esse parâmetro por default tem valor 0, e a implementação nesse caso ignora
o pacote ACK, o que faz a parte do sistema que trata da conexão em mais baixo
nível considere que o pacote se perdeu, e reenvie o SYN/ACK (até no máximo
\texttt{tcp\_synack\_retries} vezes usando exponencial backoff).

O TCP do lado do cliente por outro lado ao ver múltiplos SYN/ACK interpreta o
resultado como múltiplas perdas do seu ACK, e continua a reenviar o mesmo.

Isso e o fato do parâmetro backlog ser usado como indicação (e aparentemente
ignorado em prol de um valor significativamente maior) atrapalha bastante a
realização da questão 3, pois é necessário preencher totalmente a fila de
backlog e esperar o tempo do processo de exponencial backoff até que se exceda o
numero de retries permitidos (5), o que pode demorar mais de 100 segundos

\section{Questão 2}
\label{sec:orgheadline2}
Códigos anexos

\section{Questão 3}
\label{sec:orgheadline3}
O valor que passo como parâmetro para o tamanho do backlog não aparenta ser
utilizado para valores pequenos, utilizando \texttt{netstat -a} verifiquei que o
sistema sempre aceita no minimo 32 conexões (que ficam com estado \texttt{ESTABLISHED}
antes de começar a deixar conexões em \texttt{SYN\_RECV}, e nenhuma conexão é derrubada
por mais de 100s. 

Curiosamente após setar o parâmetro \texttt{tcp\_abort\_on\_overflow} para 1 não apenas a
as conexões pararam de ser interrompidas quando o backlog é preenchido, como o
valor passado como parâmetro passa a ser respeitado.

\section{Questão 4}
\label{sec:orgheadline4}
Ao executar o sniffer é possível ver claramente o three way handshake
acontecendo em dois tipo de conexão, as que acontecem antes do backlog ser
preenchido e tem a sequencia de flags (S, S., .) (S -> SYN, . -> ACK), e depois
do backlog encher temos (S,S.,.,R), ou seja, o sistema aceita a conexão e
imediatamente envia um RESET para encerrar a mesma.

\section{Questão 5}
\label{sec:orgheadline5}
Basta tratar o sinal SIGCHD, seja capturando o mesmo e chamando \texttt{wait()} ou da
seguinte forma:

\begin{verbatim}
struct sigaction sigchld_action = {
  .sa_handler = SIG_DFL,
  .sa_flags = SA_NOCLDWAIT
};
sigaction(SIGCHLD, &sigchld_action, NULL);
\end{verbatim}
\end{document}
