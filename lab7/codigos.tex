% Created 2015-11-10 Tue 00:37
\documentclass[11pt]{article}
\usepackage[utf8]{inputenc}
\usepackage[T1]{fontenc}
\usepackage{fixltx2e}
\usepackage{graphicx}
\usepackage{grffile}
\usepackage{longtable}
\usepackage{wrapfig}
\usepackage{rotating}
\usepackage[normalem]{ulem}
\usepackage{amsmath}
\usepackage{textcomp}
\usepackage{amssymb}
\usepackage{capt-of}
\usepackage{hyperref}
\author{Gabriel Hidasy Rezende}
\date{\today}
\title{Laboratório 7 - códigos}
\hypersetup{
 pdfauthor={Gabriel Hidasy Rezende},
 pdftitle={},
 pdfkeywords={},
 pdfsubject={},
 pdfcreator={Emacs 24.5.1 (Org mode 8.3.2)}, 
 pdflang={English}}
\begin{document}
\maketitle
\begin{verbatim}
//Servidor
#include <stdio.h>
#include <stdlib.h>
#include <errno.h>
#include <string.h>
#include <netdb.h>
#include <sys/types.h>
#include <netinet/in.h>
#include <sys/socket.h>
#include <time.h>
#include <unistd.h>
#include <arpa/inet.h>
#include <signal.h>

#define MAXDATASIZE 100

int main (int argc, char **argv) {
     int    listenfd, connfd;
     struct sockaddr_in servaddr;
     char   buf[MAXDATASIZE];
     time_t ticks;
     if (argc != 2) {
	  printf("You must specify the size of the backlog\n");
	  exit(0);
     }

     /* Disables CLDWAIT, to prevento zombies */
     struct sigaction sigchld_action = {
	  .sa_handler = SIG_DFL,
	  .sa_flags = SA_NOCLDWAIT
     };
     sigaction(SIGCHLD, &sigchld_action, NULL);
     /* It could also be done like in lab6, registering a handler for SIGCHD and
	calling wait() */


     int backlog;
     sscanf(argv[1],"%d",&backlog);
     /* No trecho abaixo um socket é inicializado */
     if ((listenfd = socket(AF_INET, SOCK_STREAM, 0)) == -1) {
	  perror("socket");
	  exit(1);
     }

     /* Estruturas são preparadas */
     bzero(&servaddr, sizeof(servaddr));
     servaddr.sin_family      = AF_INET;
     servaddr.sin_addr.s_addr = htonl(INADDR_ANY);
     servaddr.sin_port        = htons(12344);   

     /* A porta (12344) é reservada */
     if (bind(listenfd, (struct sockaddr *)&servaddr, sizeof(servaddr)) == -1) {
	  perror("bind");
	  exit(1);
     }

     /* O servidor passa a receber conexões */
     if (listen(listenfd, backlog) == -1) {
	  perror("listen");
	  exit(1);
     }

     /* Loop principal do servidor, aqui ele escuta uma conexão e ao receber uma */
     /* retorna a data do dia, note que esse servidor não é multi-thread mas */
     /* ainda funciona bem com varios clientes pois ele mesmo se encarrega de */
     /* fechar a conexão */
     for ( ; ; ) {
	  if ((connfd = accept(listenfd, (struct sockaddr *) NULL, NULL)) == -1 ) {
	       perror("accept");
	       exit(1);
	  }
	  if (fork() == 0) {
	       ticks = time(NULL);
	       struct sockaddr_in remote;
	       int len = sizeof(remote);
	       getpeername(connfd, (struct sockaddr *) &remote, &len);
	       printf("Remote socket %s\n",inet_ntoa(remote.sin_addr));

	       /* Aqui a mensagem de resposta é montada */
	       snprintf(buf, sizeof(buf), "%.24s\r\n", ctime(&ticks));
	       /* E enviada */
	       write(connfd, buf, strlen(buf));
	       /* O socket dessa conexão é terminado e vamos para o proximo cliente */
	       close(connfd);
	       exit(0);
	  }
	  close(connfd);
	  system("sleep 1024");
     }
     return(0);
}


//Cliente
#include <sys/socket.h>
#include <sys/types.h>
#include <arpa/inet.h>
#include <netinet/in.h>
#include <stdio.h>
#include <netdb.h>
#include <string.h>
#include <errno.h>
#include <string.h>
#include <stdlib.h>
#include <unistd.h>

#define MAXLINE 4096

int main(int argc, char **argv) {
   int    sockfd, n;
   char   recvline[MAXLINE + 1];
   char   error[MAXLINE + 1];
   struct sockaddr_in servaddr;

   if (argc != 2) {
      strcpy(error,"uso: ");
      strcat(error,argv[0]);
      strcat(error," <IPaddress>");
      perror(error);
      exit(1);
   }

   /* Abre o socket */
   if ( (sockfd = socket(AF_INET, SOCK_STREAM, 0)) < 0) {
      perror("socket error");
      exit(1);
   }

   /* Prepara as estruturas que serão usadas */
   bzero(&servaddr, sizeof(servaddr));
   servaddr.sin_family = AF_INET;
   servaddr.sin_port   = htons(12344);

   /* Aqui é criada a estrutura que será usada para a conexão */
   if (inet_pton(AF_INET, argv[1], &servaddr.sin_addr) <= 0) {
      perror("inet_pton error");
      exit(1);
   }

   /* A conexão é feita */
   if (connect(sockfd, (struct sockaddr *) &servaddr, sizeof(servaddr)) < 0) {
      perror("connect error");
      exit(1);
   }
   struct sockaddr_in sockname;
   /* Seus dados são impressos na tela */
   getsockname(sockfd, (struct sockaddr *) &sockname, (socklen_t *) sizeof(sockname));
   printf("local IP is %s\n",inet_ntoa(sockname.sin_addr));

   /* Dados são recebidos e impressos */
   while ( (n = read(sockfd, recvline, MAXLINE)) > 0) {
      recvline[n] = 0;
      if (fputs(recvline, stdout) == EOF) {
	 perror("fputs error");
	 exit(1);
      }
   }

   if (n < 0) {
      perror("read error");
      exit(1);
   }

   exit(0);
}
\end{verbatim}
\end{document}
