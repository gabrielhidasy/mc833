% Created 2015-10-21 Wed 00:06
\documentclass[11pt]{article}
\usepackage[utf8]{inputenc}
\usepackage[T1]{fontenc}
\usepackage{fixltx2e}
\usepackage{graphicx}
\usepackage{longtable}
\usepackage{float}
\usepackage{wrapfig}
\usepackage{rotating}
\usepackage[normalem]{ulem}
\usepackage{amsmath}
\usepackage{textcomp}
\usepackage{marvosym}
\usepackage{wasysym}
\usepackage{amssymb}
\usepackage{hyperref}
\tolerance=1000
\author{Gabriel Hidasy Rezende}
\date{\today}
\title{Atividade 6 - Servidor TCP com}
\hypersetup{
  pdfkeywords={},
  pdfsubject={},
  pdfcreator={Emacs 24.5.1 (Org mode 8.2.10)}}
\begin{document}

\maketitle


\section{Questão 1:}
\label{sec-1}
O servidor em questão nunca aceitou conexões concorrentes, mas dava a
impressão de o fazer já que a tarefa executada por cliente era
realizada muito rapido e a conexão encerrada em sequencia. 

A mudança deixou o processamento por cliente mais lento, evidenciando
o carater serial do servidor, quando dois clientes se conectam ao
mesmo tempo o primeiro recebe a resposta enquanto o segundo espera o
sleep terminar antes de receber dados.

Ambas as conexões são bem sucedidas já que temos um buffer (de tamanho
10, definido em \texttt{listen(sockfd,buffer\_size)} para coneções ainda não
aceitas.

Mas fazendo um numero grande de conexões num curto intervalo de tempo
começamos a ter falhas de conexão

\section{Questão 2}
\label{sec-2}
Utilizei os programas fornecidos como base, eles seguem anexos em
outro pdf
\section{Questão 3}
\label{sec-3}
Ainda utilizando os mesmos como base, seguem anexos em pfg
\section{Questão 4}
\label{sec-4}
Porque o no trecho apresentado o processo pai (que roda o Accept)
fecha o socket usado para se comunicar com o cliente, mas o filho
permanece com ele aberto até terminar sua operação (uma escolha
sensivel dado que o filho trata da conexão do cliente). Da mesma
forma o filho fecha o socket de listen (que ele não vai usar), mas o
pai permance com ele aberto, afinal ele trata de receber mais
clientes.
\section{Questão 5}
\label{sec-5}
Após executar o servidor iniciei 3 conexões, a saida de pstree -A o
seguinte trecho:
\begin{verbatim}
|-xfce4-terminal-+-zsh---pstree
                 |-zsh-+-3*[cliente]
                       |-servidor---3*[servidor]

\end{verbatim}

O que claramente indica que o servidor que eu rodei no meu terminal
tem 3 filhos que são instancias dele mesmo, como esperado pelos
multiplos fork

O mesmo pode ser visto na saida do \texttt{ps o pid,ppid,command}
\begin{verbatim}
PID    PPID COMMAND
16267  3765 zsh
17074 16267 ./servidor
17118 16267 ./cliente 127.0.0.1 12344
17119 17074 ./servidor
17229 16267 ./cliente 127.0.0.1 12344
17230 17074 ./servidor
17676 16267 ./cliente 127.0.0.1 12344
17677 17074 ./servidor
20224 14169 ps o pid,ppid,command
\end{verbatim}


Onde claramente temos  3 uma instancia do servidor rodando para cada
cliente, alám de uma adicional que é a que roda Accept, as 3 ultimas
instancias de servidor tem o mesmo PPID (parent PID), que é igual ao
PID da instancia principal do servidor (17074)
\section{Questão 6}
\label{sec-6}
Como era de se esperar pela implementação, em situações normais (o
servidor permanece e o cliente deixa a sessão (seja com Bye ou com
um \texttt{SIGINT}), o cliente fica com o estado \texttt{TIME\_WAIT} (testei com
netstat -a -p). Em geral a maquina que inicia o processo de
desconexão (\texttt{close(fd)}) termina nesse estado.
% Emacs 24.5.1 (Org mode 8.2.10)
\end{document}
