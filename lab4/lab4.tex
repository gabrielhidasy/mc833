% Created 2015-09-20 Sun 22:33
\documentclass[11pt]{article}
\usepackage[utf8]{inputenc}
\usepackage[T1]{fontenc}
\usepackage{fixltx2e}
\usepackage{graphicx}
\usepackage{longtable}
\usepackage{float}
\usepackage{wrapfig}
\usepackage{rotating}
\usepackage[normalem]{ulem}
\usepackage{amsmath}
\usepackage{textcomp}
\usepackage{marvosym}
\usepackage{wasysym}
\usepackage{amssymb}
\usepackage{hyperref}
\tolerance=1000
\author{Gabriel Hidasy Rezende}
\date{\today}
\title{Atividade 4 - Máquina de estados Finitos do Protocolo TCP}
\hypersetup{
  pdfkeywords={},
  pdfsubject={},
  pdfcreator={Emacs 24.5.1 (Org mode 8.2.10)}}
\begin{document}

\maketitle
Objetivo: Estudar uma implementação real da máquina de estados finitos do protocolo TCP.

\section{Questão 1:}
\label{sec-1}
Analise o arquivo \texttt{tcp\_states.h} e verifique os estados
disponiveis. Estão inclusos todos os estados descritos
anteriormente?

Logo no inicio do arquivo temos um enum com os estados
disponiveis, segue anexa:
\begin{verbatim}
enum { TCP_ESTABLISHED = 1, 
       TCP_SYN_SENT, 
	TCP_SYN_RECV,
	TCP_FIN_WAIT1, 
	TCP_FIN_WAIT2, 
	TCP_TIME_WAIT, 
	TCP_CLOSE,
	TCP_CLOSE_WAIT, 
	TCP_LAST_ACK,  
	TCP_LISTEN, 
	TCP_CLOSING, /*Now a valid state */ 
	TCP_MAX_STATES /* Leave at the end!*/ 
};
\end{verbatim}
Obviamente não vemos o estado CLOSED, que já era descrito como
fictício. Todos os outros estados estão presentes e ainda temos o
estado CLOSE que não está na tabela, mas é equivalente ao closed
logo após o fechamento (é o estado da estrutura logo antes da
limpeza).

\section{Questão 2:}
\label{sec-2}
Identifique três funções envolvidas no processo de fechamento de
uma conexão TCP tanto do lado cliente como do lado
servidor. Suponha que o fechamento é inicializado pelo cliente
com o envio de uma mensagem FIN para o servidor após ter acabado
a transmissão de todos os pacotes de uma sessão. Descreva o papel
da função no processo.

No arquivo \texttt{tcp\_output.c} temos a função \texttt{tcp\_send\_fin} que tem um
nome auto-explicativo, ela é responsavel por enviar os pacotes
FIN, ela também é responsavel por enviar os pacotes restantes no
buffer, o cliente chamaria ela para iniciar o processo de
finalizar a transmissão (\texttt{\_\_tcp\_push\_pending\_frames} com um FIN)

No arquivo \texttt{tcp\_input.c} temos a função \texttt{tcp\_fin} que é responsavel
por tratar o recebimento de um FIN, limpar a memoria usada pela
estutura sk e levar a maquina de estados para um dos estados de
encerramento.

E temos em \texttt{tcp.c} a função \texttt{tcp\_close} é usada para iniciar o
processo de finalização de uma conexão.

\section{Questão 3}
\label{sec-3}
Escolha 2 funções nos arquivos \texttt{tcp.c}, \texttt{tcp\_input.c} ou
\texttt{tcp\_output.c} que trabalhem diretamente com o estado do TCP. Pelo
menos uma das funções deve mudar o estado dentro da função. Crie
um relatório explicando a implementação das funções escolhidas.

\subsection{\texttt{tcp\_set\_state} do arquivo tcp.c}
\label{sec-3-1}
Como o proprio nome diz, essa função é reponsavel por
efetivamente modificar o estado da maquina de estados do TCP, com
algum tratamente (e coleta de estatisticas) dependendo das
transições requeridas.
Ela é estraturada como um grande switch do estado atual, uma
implementação comum para maquinas de estado (por exemplo, uma mudança
para CLOSE leva o socket a ser removido da tabela de sockets, enquanto
uma mudança de ESTABILISHED para ESTABILISHED apenas contabiliza
estatisticas.

\subsection{\texttt{tcp\_fin} do \texttt{tcp\_input.c}}
\label{sec-3-2}
Essa função, já vista na questão 2, é responsavel por tratar o
recebimento de pacotes com a flag FIN.
Ela aje sobre a maquina de estados da seguinte forma:

\begin{itemize}
\item se estava \texttt{ESTABILISHED} ou \texttt{SYN\_RECV} vai para \texttt{CLOSE\_WAIT}

\item se estava em \texttt{CLOSE\_WAIT} ou \texttt{CLOSING} assume que é
retransmissão do \texttt{FIN} e não faz nada

\item se estava em \texttt{LAST\_ACK} não faz nada de acordo com a RFC793

\item se estava em \texttt{FIN\_WAIT1} enviar um \texttt{ACK} pelo \texttt{FIN} e vai para \texttt{CLOSING}

\item se estava em \texttt{FIN\_WAIT2} enviar um \texttt{ACK} e entrar em \texttt{TIME\_WAIT}.
\end{itemize}

Em seguida ela programa o shutdown do socket e termina
% Emacs 24.5.1 (Org mode 8.2.10)
\end{document}
