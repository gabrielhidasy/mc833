% Created 2015-10-04 Sun 21:36
\documentclass[11pt]{article}
\usepackage[utf8]{inputenc}
\usepackage[T1]{fontenc}
\usepackage{fixltx2e}
\usepackage{graphicx}
\usepackage{longtable}
\usepackage{float}
\usepackage{wrapfig}
\usepackage{rotating}
\usepackage[normalem]{ulem}
\usepackage{amsmath}
\usepackage{textcomp}
\usepackage{marvosym}
\usepackage{wasysym}
\usepackage{amssymb}
\usepackage{hyperref}
\tolerance=1000
\usepackage{listings}
\usepackage{color}

\definecolor{dkgreen}{rgb}{0,0.6,0}
\definecolor{gray}{rgb}{0.5,0.5,0.5}
\definecolor{mauve}{rgb}{0.58,0,0.82}

\lstset{frame=tb,
  language=C,
  aboveskip=3mm,
  belowskip=3mm,
  showstringspaces=false,
  columns=flexible,
  basicstyle={\small\ttfamily},
  numbers=none,
  numberstyle=\tiny\color{gray},
  keywordstyle=\color{blue},
  commentstyle=\color{dkgreen},
  stringstyle=\color{mauve},
  breaklines=true,
  breakatwhitespace=true,
  tabsize=3
}
\author{Gabriel Hidasy Rezende}
\date{\today}
\title{Atividade 5 - Implementação de Aplicação Cliente/Servidor com Sockets TCP -
  Códigos}
\hypersetup{
  pdfkeywords={},
  pdfsubject={},
  pdfcreator={Emacs 24.5.1 (Org mode 8.2.10)}}
\begin{document}

\maketitle

\begin{lstlisting}
  //Cliente.C
  #include <sys/socket.h>
  #include <sys/types.h>
  #include <arpa/inet.h>
  #include <netinet/in.h>
  #include <stdio.h>
  #include <netdb.h>
  #include <string.h>
  #include <errno.h>
  #include <string.h>
  #include <stdlib.h>
  #include <unistd.h>

  #define MAXLINE 4096

  int main(int argc, char **argv) {
    int    sockfd, n;
    char   recvline[MAXLINE + 1];
    char   error[MAXLINE + 1];
    struct sockaddr_in servaddr;

    if (argc != 2) {
      strcpy(error,"uso: ");
      strcat(error,argv[0]);
      strcat(error," <IPaddress>");
      perror(error);
      exit(1);
    }

    /* Abre o socket */
    if ( (sockfd = socket(AF_INET, SOCK_STREAM, 0)) < 0) {
      perror("socket error");
      exit(1);
    }

    /* Prepara as estruturas que serao usadas */
    bzero(&servaddr, sizeof(servaddr));
    servaddr.sin_family = AF_INET;
    servaddr.sin_port   = htons(12344);

    /* Aqui eh criada a estrutura que sera usada para a conexao */
    if (inet_pton(AF_INET, argv[1], &servaddr.sin_addr) <= 0) {
      perror("inet_pton error");
      exit(1);
    }

    /* A conexao eh feita */
    if (connect(sockfd, (struct sockaddr *) &servaddr, sizeof(servaddr)) < 0) {
      perror("connect error");
      exit(1);
    }
    struct sockaddr_in sockname;
    /* Seus dados sao impressos na tela */
    getsockname(sockfd, (struct sockaddr *) &sockname, (socklen_t *) sizeof(sockname));
    printf("local IP is %s\n",inet_ntoa(sockname.sin_addr));

    /* Dados sao recebidos e impressos */
    while ( (n = read(sockfd, recvline, MAXLINE)) > 0) {
      recvline[n] = 0;
      if (fputs(recvline, stdout) == EOF) {
        perror("fputs error");
        exit(1);
      }
    }

    if (n < 0) {
      perror("read error");
      exit(1);
    }
    
    exit(0);
  }

\end{lstlisting}

\begin{lstlisting}
  //Servidor.c
  #include <stdio.h>
  #include <stdlib.h>
  #include <errno.h>
  #include <string.h>
  #include <netdb.h>
  #include <sys/types.h>
  #include <netinet/in.h>
  #include <sys/socket.h>
  #include <time.h>
  #include <unistd.h>
  #include <arpa/inet.h>

  #define LISTENQ 10
  #define MAXDATASIZE 100

  int main (int argc, char **argv) {
    int    listenfd, connfd;
    struct sockaddr_in servaddr;
    char   buf[MAXDATASIZE];
    time_t ticks;

    /* No trecho abaixo um socket eh inicializado */
    if ((listenfd = socket(AF_INET, SOCK_STREAM, 0)) == -1) {
      perror("socket");
      exit(1);
    }

    /* Estruturas sao preparadas */
    bzero(&servaddr, sizeof(servaddr));
    servaddr.sin_family      = AF_INET;
    servaddr.sin_addr.s_addr = htonl(INADDR_ANY);
    servaddr.sin_port        = htons(12344);   

    /* A porta (12344) e reservada */
    if (bind(listenfd, (struct sockaddr *)&servaddr, sizeof(servaddr)) == -1) {
      perror("bind");
      exit(1);
    }

    /* O servidor passa a receber conexoes */
    if (listen(listenfd, LISTENQ) == -1) {
      perror("listen");
      exit(1);
    }

    /* Loop principal do servidor, aqui ele escuta uma conexao e ao receber uma */
    /* retorna a data do dia, note que esse servidor nao eh multi-thread mas */
    /* ainda funciona bem com varios clientes pois ele mesmo se encarrega de */
    /* fechar a conexao */
    for ( ; ; ) {
      if ((connfd = accept(listenfd, (struct sockaddr *) NULL, NULL)) == -1 ) {
        perror("accept");
        exit(1);
      }

      ticks = time(NULL);
      struct sockaddr_in remote;
      int len = sizeof(remote);
      getpeername(connfd, (struct sockaddr *) &remote, &len);
      printf("Remote socket %s\n",inet_ntoa(remote.sin_addr));

      /* Aqui a mensagem de resposta eh montada */
      snprintf(buf, sizeof(buf), "%.24s\r\n", ctime(&ticks));
      /* E enviada */
      write(connfd, buf, strlen(buf));
      /* O socket dessa conexao eh terminado e vamos para o proximo cliente */
      close(connfd);
    }
    return(0);
  }

\end{lstlisting}
\end{document}
