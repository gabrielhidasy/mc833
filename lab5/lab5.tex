% Created 2015-10-04 Sun 21:36
\documentclass[11pt]{article}
\usepackage[utf8]{inputenc}
\usepackage[T1]{fontenc}
\usepackage{fixltx2e}
\usepackage{graphicx}
\usepackage{longtable}
\usepackage{float}
\usepackage{wrapfig}
\usepackage{rotating}
\usepackage[normalem]{ulem}
\usepackage{amsmath}
\usepackage{textcomp}
\usepackage{marvosym}
\usepackage{wasysym}
\usepackage{amssymb}
\usepackage{hyperref}
\tolerance=1000
\usepackage{listings}
\usepackage{color}

\definecolor{dkgreen}{rgb}{0,0.6,0}
\definecolor{gray}{rgb}{0.5,0.5,0.5}
\definecolor{mauve}{rgb}{0.58,0,0.82}

\lstset{frame=tb,
  language=C,
  aboveskip=3mm,
  belowskip=3mm,
  showstringspaces=false,
  columns=flexible,
  basicstyle={\small\ttfamily},
  numbers=none,
  numberstyle=\tiny\color{gray},
  keywordstyle=\color{blue},
  commentstyle=\color{dkgreen},
  stringstyle=\color{mauve},
  breaklines=true,
  breakatwhitespace=true,
  tabsize=3
}
\author{Gabriel Hidasy Rezende}
\date{\today}
\title{Atividade 5 - Implementação de Aplicação Cliente/Servidor com Sockets TCP}
\hypersetup{
  pdfkeywords={},
  pdfsubject={},
  pdfcreator={Emacs 24.5.1 (Org mode 8.2.10)}}
\begin{document}

\maketitle

\section*{Questão 1:}
\label{sec-1}

\subsection*{cliente.c}
\label{sec-1-1}

As funções interessantes da parte de redes são:

\begin{itemize}
\item \texttt{int socket(int domain, int type, int protocol)}
\end{itemize}
Requisita um socket para o sistema operacional, retorna um descritor
de arquivo (o socket nada mais é do que um arquivo em sistemas unix),
retorna o socket ou -1 em caso de erro

\begin{itemize}
\item \texttt{int connect(int sockfd, const struct sockaddr *addr,socklen\_t addrlen)}
\end{itemize}
Usa a system call connect para, como o nome diz, conectar o socket a
um endereço (especificado na estrutura sockaddr).
No caso desse programa (que usa sockets do tipo stream) isso significa
realmente realizar uma conexão TCP, realizando o three way handshake
com o servidor especificado em sockaddr.
Se o socket fosse do tipo DGRAM ela iria especificar para que endereço
enviar (e de que endereço aceitar receber) datagramas.
retorna 0 em caso de sucesso e -1 em caso de erro

\begin{itemize}
\item \texttt{int read(int sockfd, char *data, int buffer\_size)}
\end{itemize}
Lê dados do socket e os armazena em \texttt{=data=}, o valor de retorno
indica quantos bytes foram lidos do socket especificado.

\begin{itemize}
\item \texttt{int inet\_pton(int af, const char *src, void *dst)}
\end{itemize}
Essa função converte uma string de caracteres (src) em uma estrutura
do tipo especificado em af (AF$_{\text{INET}}$ ou AF$_{\text{INET6}}$), e copia esse
endereço para dst
Retorna 1 em caso de sucesso, 0 se src não contem um endereço do tipo
especificado em af, e -1 se af for de um tipo não suportado

\begin{itemize}
\item \texttt{uint16\_t htons(uint16\_t hostname)}
\end{itemize}
Essa função converte um unsigned short int de host-byte-order para
network-byte-order, preparando o endereço para ser usado na rede 

\subsection{servidor.c}
\label{sec-1-2}

\begin{itemize}
\item \texttt{int bind(int sockfd, struct sockaddr * servaddr, int size)}
\end{itemize}
Liga um socket a uma porta (especificada em servaddr.sin$_{\text{port}}$), é
usada no servidor para permitir que ele ouça em uma porta especifica
por clientes.

\begin{itemize}
\item \texttt{int listen(int sockfd, int backlog)}
\end{itemize}
Marca um socket como um socket como passivo, o que significa que ele
será usado para receber conexões, backlog diz o tamanho da fila de
conexões pendentes que será aceito.
Retorna 0 em caso de sucesso, e -1 em caso de erro

\begin{itemize}
\item \texttt{int accept(int sockfd, struct sockaddr * servaddr, int size)}
\end{itemize}
Aceita conexões, retorna um file descriptor para um socket ligado ao
cliente. A thread que roda essa chamada fica bloqueada até que um
cliente se conecte.

\begin{itemize}
\item \texttt{uint32\_t htonl(uint32\_t hostname)}
\end{itemize}
Faz o mesmo que a htons, mas com inteiros de 32 bits
São simples conversores de big-ending-little-ending

\section*{Questão 2:}
\label{sec-2}
A função bind no servidor retorna um erro pois ela está tentando ligar
o socket a porta 13, que é uma porta baixa (por padrão apenas o super
usuário pode usar bind em portas abaixo de 1024), mudando o servidor
para usar uma porta alta (ex. 12345) resolve o problema.
Obviamente nesse caso precisamos modificar o cliente para que ele se
conecte na porta correta.
Feitas as modificações e executando ambos os comandos na mesma
maquina, e passando o endereço local para o cliente (./cliente
127.0.0.1), temos como saída a data e hora de hoje (Mon Sep 28
22:20:53 2015). O servidor não imprime nada

\section*{Questão 3}
\label{sec-3}
Executando em maquinas diferentes tive o mesmo resultado, só precisei
usar o ip da outra maquina (./cliente 177.220.85.187). (Na outra
maquina pude user a porta 13)

\section*{Questão 4}
\label{sec-4}
Usando o tcpdump chequei o trafego passando pelo meu notebook, mais
especificamente com o comando \texttt{sudo tcpdump -A | grep Mon}, e a cada
requisição do cliente o tcpdump registrava mais um pacote com a data
de hoje sendo enviada. Também é possivel usar telnet ou netcat como
cliente (ambas as ferramentas não geram a saida com a data por si só,
o que significa que a string de data tem que ter vindo do servidor)

\section*{Questão 5}
\label{sec-5}
No código

\section*{Questão 6}
\label{sec-6}
No código, após as modificações e rodando em duas maquinas diferentes
tive a saída que segue:

\begin{verbatim}
./servidor
Remote socket 192.168.1.212

./cliente 192.168.1.147
local IP is 0.0.0.0
Sun Oct  4 20:21:49 2015
\end{verbatim}

A resposta com IP local zerada era esperada, pois getsockename retorna
com que IP o socket está ligado (bound), e um socket de cliente em
geral não está bound a nenhum IP especifico (0.0.0.0 significa
"Qualquer endereço" - \texttt{INADDR\_ANY})

\section*{Questão 7:}
\label{sec-7}

A maquina servidor em estado \texttt{TIME\_WAIT}, que é o ultimo estado em que
a maquina que iniciou a desconexão fica, isso acontece pois o servidor
é a primeira maquina a fechar a conexão (\verb~close(connfd)~), em geral
não é uma boa ideia deixar o servidor em \texttt{TIME\_WAIT} pois ele é mais
sujeito (por ter vários clientes) a ficar sem recursos com múltiplas
conexões nesse estado.

\section*{Questão 8}
\label{sec-8}

Sim, telnet pode ser usado para no lugar do cliente, basta rodar o
comando telnet especificando a porta correta (no meu caso 12344).

\begin{verbatim}
telnet 192.168.1.147 12344
Trying 192.168.1.212...
Connected to 192.168.1.212.
Escape character is '^]'.
Mon Oct  5 00:00:38 2015
Connection closed by foreign host.

./servidor
Remote socket 192.168.1.125
\end{verbatim}

Para o servidor não faz diferença qual o cliente usado.

Uma modificação no servidor que impediria o uso de comando telnet
seria modificar o mesmo para codificar as mensagens de alguma forma
(por exemplo, transmitindo os dados em Base 64, ou comprimir os dados
de alguma forma, ou cifrar eles, ou no caso do protocolo usado (que
envia datas) enviar apenas o epoch e deixar a tradução para uma data
por conta do cliente. Todas as modificações exigiriam uma mudança
adequada do lado do cliente (telnet ou netcat ainda conseguiriam
receber a string do servidor, mas a saída seria incompreensível)
% Emacs 24.5.1 (Org mode 8.2.10)

\end{document}
